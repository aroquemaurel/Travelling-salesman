\documentclass[a4paper, 11pt]{article}

\usepackage{xcolor}
\input{/home/aroquemaurel/cours/includesLaTeX/couleurs.tex}
\usepackage{lmodern}
\usepackage[utf8]{inputenc}
\usepackage[T1]{fontenc}
\usepackage[francais]{babel}
\usepackage[top=1.7cm, bottom=1.7cm, left=2.5cm, right=2.5cm]{geometry}
\usepackage{verbatim}
\usepackage{tikz} %Vectoriel
\usepackage{listings}
\usepackage{fancyhdr}
\usepackage{multido}
\usepackage{amssymb}
\usepackage{multicol}
\usepackage{float}
\usepackage[urlbordercolor={1 1 1}, linkbordercolor={1 1 1}, linkcolor=vert1, urlcolor=bleu, colorlinks=true]{hyperref}

\newcommand{\titre}{Le problème du voyageur de commerce}
\newcommand{\numero}{}
\newcommand{\typeDoc}{Projet}
\newcommand{\module}{Algorithmique}
\newcommand{\sigle}{algo}
\newcommand{\semestre}{3}

\input{/home/satenske/cours/includesLaTeX/l2/tddm.tex}
\input{/home/aroquemaurel/cours/includesLaTeX/listings.tex} %prise en charge du langage C 
\input{/home/aroquemaurel/cours/includesLaTeX/remarquesExempleAttention.tex}
\input{/home/aroquemaurel/cours/includesLaTeX/polices.tex}
\input{/home/aroquemaurel/cours/includesLaTeX/affichageChapitre.tex}
\makeatother
\begin{document}
	\maketitle
	\section{Le projet}
	\subsection{Compilation}
	\begin{itemize}
		\item \texttt{make} ou \texttt{make build} Compile le projet
		\item \texttt{make clean} supprime les fichiers binaires (.o)
	\end{itemize}
	Une fois un make effectué le fichier executable est disponible en racine du projet, il se nomme \texttt{voyageurDeCommerce}. 	
	\subsection{Execution}
	Comme demandé dans le cahier des charges, le programme doit respecter une liste d'argument précis, exemples d'utilisation :
	\begin{lstlisting}[language=sh, basicstyle=\scriptsize\ttfamily, caption=Exemple d'execution du programme]
./voyageurDeCommerce -v -f resources/inputFiles/essai8.txt -bf
./voyageurDeCommerce -f resources/inputFiles/essai8.txt -bf -lsr 50
./voyageurDeCommerce -f resources/inputFiles/ulysse16.txt -lsnr 20 -lsr 50
./voyageurDeCommerce -v -f resources/inputFiles/ulysse16.txt -ga 15 0.8
	\end{lstlisting}
	\subsection{Organisation des fichiers}
	Afin d'avoir une meilleur clarté, le projet est organisé en plusieurs fichiers, qui sont répartis dans différents dossiers, ci-dessous l'utilité de chacun
	des dossiers.

	\begin{description}
		\item[{build}] Ce dossier contient les fichiers binaires (.o) du projet, ceux-ci seront supprimés en utilisant la commande
			\texttt{make clean} 
		\item[{doc}] Ce dossier contient la documentation générée à l'aide de \texttt{doxygen} du projet.\footnote{Celle-ci est disponible en format \bsc{HTML} ou
			pdf.} \\La documentation du projet est également disponible à l'adresse suivante : \\ \lien{http://documentation.joohoo.fr/L2/voyageurCommerce/}
		\item[{lib}] Ce dossier contient les fichiers headers (.h)
		\item[{src}] Ce dossier contient les sources du projet (.c) 
		\item[{resources}] Ici sont entreposés les fichiers de ressources pouvant être utiles au programmes 
		\item[{tests}] Les fichiers de tests. 
	\end{description}
	\section{Modifications apportés depuis la validation du 7 Janvier 2013}
	Les modifications depuis la validation : 
	\begin{itemize}
		\item Correction bug des recherches locales.
	\end{itemize}
	\section{Programmation avec preuve de \texttt{util\_searchFirstOccurence}}
\end{document}
